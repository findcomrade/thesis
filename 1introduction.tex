\chapter{Introduction}
\label{chapter:intro}

High-Throughput Drug Testing is a technology that allows to reveal the impact of a particular drug (set of multiple drugs) on an individual. Sample is taken from patients and then cell line cultured from the tissue affected by cancer. After the specimens have been incorporated on a plate, cell viability is measured and serves as a key value to assess drug response. While methodology of the biological experiment quite similar to RNAi or Small Molecules Screening, its computational side brings new challenges when identifying new drug sets. Here drug set implies multiple drugs that cause similar response against single target (cancer sample).  In other words, Drug Set Enrichment Analysis intends to facilitate the assessment of drug responses across multiple cell lines and also allow for drug response prediction. The figure below shows how cell lines from different cancer types respond to treatment with Nilotinib, CYT387 and EKB.569 drugs.

A Figure goes here...




\section{Problem statement}

We can see different drug response from the picture, which means that some drugs perform better on a certain condition than the others. Thus, a set of drugs can be identified to act efficiently when curing a particular disease. Consequently, the first task would be to apply clustering to the samples in order to identify those drug sets. Then, once a new drug comes it should be classified and assigned to one of the existing clusters, which refers to Drug Set Enrichment Analysis. So, drug response prediction and classification appear as the second challenge in the project.

\section{Helpful hints}

The project will be conducted within the group of Individualized Systems Medicine in Institute for Molecular Medicine Finland FIMM.

\section{Structure of the Thesis}
\label{section:structure} 

You should use transition in your text, meaning that you should help
the reader follow the thesis outline. Here, you tell what will be in
each chapter of your thesis. 

