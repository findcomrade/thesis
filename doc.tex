% Compilation instructions: 
% ------------------------------------------------------------------
% Use pdflatex to compile! Input images are expected as PDF files.
% Example compilation:
% ------------------------------------------------------------------
% > pdflatex thesis-example.tex
% > bibtex thesis-example
% > pdflatex thesis-example.tex
% > pdflatex thesis-example.tex
% ------------------------------------------------------------------
% You need to run pdflatex multiple times so that all the cross-references
% are fixed. pdflatex will tell you if you need to re-run it (a warning
% will be issued)  
% General information
% ==================================================================
% Document class for the thesis is report
% ------------------------------------------------------------------
% You can change this but do so at your own risk - it may break other things.
% Note that the option pdftext is used for pdflatex; there is no
% pdflatex option. 
% ------------------------------------------------------------------
\documentclass[12pt,a4paper,oneside,pdftex]{report}

% The input files (tex files) are encoded with the latin-1 encoding 
% (ISO-8859-1 works). Change the latin1-option if you use UTF8 
% (at some point LaTeX did not work with UTF8, but I'm not sure
% what the current situation is) 
\usepackage[latin1]{inputenc}
% OT1 font encoding seems to work better than T1. Check the rendered
% PDF file to see if the fonts are encoded properly as vectors (instead
% of rendered bitmaps). You can do this by zooming very close to any letter 
% - if the letter is shown pixelated, you should change this setting 
% (try commenting out the entire line, for example).  
\usepackage[OT1]{fontenc}
% The babel package provides hyphenating instructions for LaTeX. Give
% the languages you wish to use in your thesis as options to the babel
% package (as shown below). You can remove any language you are not
% going to use.
% Examples of valid language codes: english (or USenglish), british, 
% finnish, swedish; and so on.
\usepackage[finnish,swedish,english]{babel}
% Optional packages
% ------------------------------------------------------------------
% Select those packages that you need for your thesis. You may delete
% or comment the rest.

% Natbib allows you to select the format of the bibliography references.
% The first example uses numbered citations: 
\usepackage[square,sort&compress,numbers]{natbib}
% The second example uses author-year citations.
% If you use author-year citations, change the bibliography style (below); 
% acm style does not work with author-year citations.
% Also, you should use \citet (cite in text) when you wish to refer
% to the author directly (\citet{blaablaa} said blaa blaa), and 
% \citep when you wish to refer similarly than with numbered citations
% (It has been said that blaa blaa~\citep{blaablaa}).
% \usepackage[square]{natbib}

% The alltt package provides an all-teletype environment that acts
% like verbatim but you can use LaTeX commands in it. Uncomment if 
% you want to use this environment. 
% \usepackage{alltt}

% The eurosym package provides a euro symbol. Use with \euro{}
\usepackage{eurosym} 

% Verbatim provides a standard teletype environment that renderes
% the text exactly as written in the tex file. Useful for code
% snippets (although you can also use the listings package to get
% automatic code formatting). 
\usepackage{verbatim}

% The listing package provides automatic code formatting utilities
% so that you can copy-paste code examples and have them rendered
% nicely. See the package documentation for details.
% \usepackage{listings}

% The fancuvrb package provides fancier verbatim environments 
% (you can, for example, put borders around the verbatim text area
% and so on). See package for details.
\usepackage{fancyvrb}

% Supertabular provides a tabular environment that can span multiple 
% pages. 
%\usepackage{supertabular}
% Longtable provides a tabular environment that can span multiple 
% pages. This is used in the example acronyms file. 
\usepackage{longtable}

% The fancyhdr package allows you to set your the page headers 
% manually, and allows you to add separator lines and so on. 
% Check the package documentation. 
% \usepackage{fancyhdr}

% Subfigure package allows you to use subfigures (i.e. many subfigures
% within one figure environment). These can have different labels and
% they are numbered automatically. Check the package documentation. 
\usepackage{subfigure}

% The titlesec package can be used to alter the look of the titles 
% of sections, chapters, and so on. This example uses the ``medium'' 
% package option which sets the titles to a medium size, making them
% a bit smaller than what is the default. You can fine-tune the 
% title fonts and sizes by using the package options. See the package
% documentation.
\usepackage[medium]{titlesec}

% The TikZ package allows you to create professional technical figures.
\usepackage{tikz}
% You also need to specify which TikZ libraries you use
\usetikzlibrary{positioning}
\usetikzlibrary{calc}
\usetikzlibrary{arrows}
\usetikzlibrary{decorations.pathmorphing,decorations.markings}
\usetikzlibrary{shapes}
\usetikzlibrary{patterns}


% The aalto-thesis package provides typesetting instructions for the
% standard master's thesis parts (abstracts, front page, and so on)
% Load this package second-to-last, just before the hyperref package.
% Options that you can use: 
%   mydraft - renders the thesis in draft mode. 
%             Do not use for the final version. 
%   doublenumbering - [optional] number the first pages of the thesis
%                     with roman numerals (i, ii, iii, ...); and start
%                     arabic numbering (1, 2, 3, ...) only on the 
%                     first page of the first chapter
%   twoinstructors  - changes the title of instructors to plural form
%   twosupervisors  - changes the title of supervisors to plural form
\usepackage[mydraft,doublenumbering,twosupervisors]{aalto-thesis}
%\usepackage[mydraft,doublenumbering]{aalto-thesis}
%\usepackage{aalto-thesis}


% Hyperref
% ------------------------------------------------------------------
% Hyperref creates links from URLs, for references, and creates a
% TOC in the PDF file.
% This package must be the last one you include, because it has
% compatibility issues with many other packages and it fixes
% those issues when it is loaded.   
\RequirePackage[pdftex]{hyperref}
% Setup hyperref so that links are clickable but do not look 
% different
\hypersetup{colorlinks=false,raiselinks=false,breaklinks=true}
\hypersetup{pdfborder={0 0 0}}
\hypersetup{bookmarksnumbered=true}
% The following line suggests the PDF reader that it should show the 
% first level of bookmarks opened in the hierarchical bookmark view. 
\hypersetup{bookmarksopen=true,bookmarksopenlevel=1}
% Hyperref can also set up the PDF metadata fields. These are
% set a bit later on, after the thesis setup.   


% Thesis setup
% ==================================================================
% Change these to fit your own thesis.
% \COMMAND always refers to the English version;
% \FCOMMAND refers to the Finnish version; and
% \SCOMMAND refers to the Swedish version.
% You may comment/remove those language variants that you do not use
% (but then you must not include the abstracts for that language)
% ------------------------------------------------------------------
% If you do not find the command for a text that is shown in the cover page or
% in the abstract texts, check the aalto-thesis.sty file and locate the text
% from there. 
% All the texts are configured in language-specific blocks (lots of commands
% that look like this: \renewcommand{\ATCITY}{Espoo}.
% You can just fix the texts there. Just remember to check all the language
% variants you use (they are all there in the same place). 
% ------------------------------------------------------------------
\newcommand{\TITLE}{Drug Set Enrichment Analysis (DSEA): }
\newcommand{\SUBTITLE}{A Computational Approach to Identify \\
Functional Drug Sets from High-Throughput Drug Testing}
\newcommand{\DATE}{August 30, 2013}

% Supervisors and instructors
% ------------------------------------------------------------------
% If you have two supervisors, write both names here, separate them with a 
% double-backslash (see below for an example)
% Also remember to add the package option ``twosupervisors'' or
% ``twoinstructors'' to the aalto-thesis package so that the titles are in
% plural.
% Example of one supervisor:
%\newcommand{\SUPERVISOR}{Professor Antti Yl�-J��ski}
%\newcommand{\FSUPERVISOR}{Professori Antti Yl�-J��ski}
%\newcommand{\SSUPERVISOR}{Professor Antti Yl�-J��ski}
% Example of twosupervisors:
\newcommand{\SUPERVISOR}{Professor Harri L�hdesm�ki}

% If you have only one instructor, just write one name here
\newcommand{\INSTRUCTOR}{Olli Ohjaaja M.Sc. (Tech.)}
% If you have two instructors, separate them with \\ to create linefeeds
% \newcommand{\INSTRUCTOR}{Olli Ohjaaja M.Sc. (Tech.)\\
%  Elli Opas M.Sc. (Tech)}
%\newcommand{\FINSTRUCTOR}{Diplomi-insin��ri Olli Ohjaaja\\
%  Diplomi-insin��ri Elli Opas}
%\newcommand{\SINSTRUCTOR}{Diplomingenj�r Olli Ohjaaja\\
%  Diplomingenj�r Elli Opas}

% If you have two supervisors, it is common to write the schools
% of the supervisors in the cover page. If the following command is defined,
% then the supervisor names shown here are printed in the cover page. Otherwise,
% the supervisor names defined above are used.
\newcommand{\COVERSUPERVISOR}{Professor Harri L�hdesm�ki, Aalto University School of Science}

% The same option is for the instructors, if you have multiple instructors.
% \newcommand{\COVERINSTRUCTOR}{Olli Ohjaaja M.Sc. (Tech.), Aalto University\\
%  Elli Opas M.Sc. (Tech), Aalto SCI}


% Other stuff
% ------------------------------------------------------------------
\newcommand{\PROFESSORSHIP}{Bioinformatics}
% Professorship code ??!
\newcommand{\PROFCODE}{T-110?}
\newcommand{\KEYWORDS}{bioinformatics, computational biomedicine, cancer research, drug testing,
drug response prediction, (drug) clustering, (drug) enrichment, (drug) annotations}
\newcommand{\LANGUAGE}{English}

% AUTHOR
\newcommand{\AUTHOR}{Dmitrii Bychkov}

% Currently the English versions are used for the PDF file metadata
\hypersetup{pdftitle={\TITLE\ \SUBTITLE}} % Set the PDF title
\hypersetup{pdfauthor={\AUTHOR}} % Set the PDF author
\hypersetup{pdfkeywords={\KEYWORDS}} % Set the PDF keywords
\hypersetup{pdfsubject={Master's Thesis}} % Set the PDF subject


% Layout settings
% ------------------------------------------------------------------

% When you write in English, you should use the standard LaTeX 
% paragraph formatting: paragraphs are indented, and there is no 
% space between paragraphs.

% Use this to control how much space there is between each line of text.
% 1 is normal (no extra space), 1.3 is about one-half more space, and
% 1.6 is about double line spacing.  
% \linespread{1} % This is the default

% Bibliography style
% acm style gives you a basic reference style. It works only with numbered
% references.
\bibliographystyle{acm}
% Plainnat is a plain style that works with both numbered and name citations.
% \bibliographystyle{plainnat}

% Extra hyphenation settings
% ------------------------------------------------------------------
% You can list here all the files that are not hyphenated correctly.
% You can provide many \hyphenation commands and/or separate each word
% with a space inside a single command. Put hyphens in the places where
% a word can be hyphenated.
% Note that (by default) LaTeX will not hyphenate words that already
% have a hyphen in them (for example, if you write ``structure-modification 
% operation'', the word structure-modification will never be hyphenated).
% You need a special package to hyphenate those words.
\hyphenation{di-gi-taa-li-sta yksi-suun-tai-sta}


%%%%%%%%%%%%%%%%%%%%%%%%%%%%%%%%%%%%%%%%%%%%%%%%%%%%%%%%%%%%%%%%%%%%%%%
%%%%%%%%%%%%%%%%%%%%%%%%%%%%%%%%%%%%%%%%%%%%%%%%%%%%%%%%%%%%%%%%%%%%%%%
%%%%%%%%%%%%%%%%%%%%%%%%%%%%%%%%%%%%%%%%%%%%%%%%%%%%%%%%%%%%%%%%%%%%%%%
%%%%%%% The preamble ends here, and the document begins. %%%%%%%%%%%%%%

\begin{document}
% This command adds a PDF bookmark to the cover page. You may leave
% it out if you don't like it...
\pdfbookmark[0]{Cover page}{bookmark.0.cover}
% This command is defined in aalto-thesis.sty. It controls the page 
% numbering based on whether the doublenumbering option is specified
\startcoverpage

% Cover page
% ------------------------------------------------------------------
% Options: finnish, english, and swedish
% These control in which language the cover-page information is shown
\coverpage{english}


% Abstract in English
% ------------------------------------------------------------------
\thesisabstract{english}{
A dissertation or thesis is a document submitted in support of candidature
for a degree or professional qualification presenting the author's research and
findings. In some countries/universities, the word thesis or a cognate is used
as part of a bachelor's or master's course, while dissertation is normally
applied to a doctorate, whilst, in others, the reverse is true.

\fixme{Abstract text goes here (and this is an example how to use fixme).} 
Fixme is a command that helps you identify parts of your thesis that still
require some work. When compiled in the custom \texttt{mydraft} mode, text
parts tagged with fixmes are shown in bold and with fixme tags around them. When
compiled in normal mode, the fixme-tagged text is shown normally (without
special formatting). The draft mode also causes the ``Draft'' text to appear on
the front page, alongside with the document compilation date. The custom
\texttt{mydraft} mode is selected by the \texttt{mydraft} option given for the
package \texttt{aalto-thesis}, near the top of the \texttt{thesis-example.tex}
file.

The thesis example file (\texttt{thesis-example.tex}), all the chapter content
files (\texttt{1introduction.tex} and so on), and the Aalto style file
(\texttt{aalto-thesis.sty}) are commented with explanations on how the Aalto
thesis works. The files also contain some examples on how to customize various
details of the thesis layout, and of course the example text works as an
example in itself. Please read the comments and the example text; that should
get you well on your way!}

% Abstract in Finnish
% Abstract in Swedish

% Acknowledgements
% ------------------------------------------------------------------
% Select the language you use in your acknowledgements
\selectlanguage{english}

% Uncomment this line if you wish acknoledgements to appear in the 
% table of contents
\addcontentsline{toc}{chapter}{Acknowledgements}

% The star means that the chapter isn't numbered and does not 
% show up in the TOC
\chapter*{Acknowledgements}

I wish to thank all students who use \LaTeX\ for formatting their theses,
because theses formatted with \LaTeX\ are just so nice.

Thank you, and keep up the good work!
\vskip 10mm

\noindent Espoo, \DATE
\vskip 5mm
\noindent\AUTHOR

% Acronyms
% ------------------------------------------------------------------
% Use \cleardoublepage so that IF two-sided printing is used 
% (which is not often for masters theses), then the pages will still
% start correctly on the right-hand side.
% \cleardoublepage
% Example acronyms are placed in a separate file, acronyms.tex
\addcontentsline{toc}{chapter}{Abbreviations and Acronyms}
\chapter*{Abbreviations and Acronyms}

% The longtable environment should break the table properly to multiple pages, 
% if needed

\noindent
\begin{longtable}{@{}p{0.25\textwidth}p{0.7\textwidth}@{}}
DSRT & Drug Sensitivity and Resistance Testing \\
IC50 & 50\% Inhibition Concentration \\ 

\end{longtable}

% Table of contents
% ------------------------------------------------------------------
%\cleardoublepage
% This command adds a PDF bookmark that links to the contents.
% You can use \addcontentsline{} as well, but that also adds contents
% entry to the table of contents, which is kind of redundant.
% The text ``Contents'' is shown in the PDF bookmark. 
\pdfbookmark[0]{Contents}{bookmark.0.contents}
\tableofcontents

% List of tables
% ------------------------------------------------------------------
% You only need a list of tables for your thesis if you have very 
% many tables. If you do, uncomment the following two lines.
% \cleardoublepage
% \listoftables

% Table of figures
% ------------------------------------------------------------------
% You only need a list of figures for your thesis if you have very 
% many figures. If you do, uncomment the following two lines.
% \cleardoublepage
% \listoffigures

% The following label is used for counting the prelude pages
\label{pages-prelude}
%\cleardoublepage

%%%%%%%%%%%%%%%%% The main content starts here %%%%%%%%%%%%%%%%%%%%%
% ------------------------------------------------------------------
% This command is defined in aalto-thesis.sty. It controls the page 
% numbering based on whether the doublenumbering option is specified
\startfirstchapter

% Add headings to pages (the chapter title is shown)
\pagestyle{headings}

% The contents of the thesis are separated to their own files.
% Edit the content in these files, rename them as necessary.
% ------------------------------------------------------------------
\chapter{Introduction}
\label{chapter:intro}

High-Throughput Drug Testing is a technology that allows to reveal the impact of a particular drug (set of multiple drugs) on an individual. Sample is taken from patients and then cell line cultured from the tissue affected by cancer. After the specimens have been incorporated on a plate, cell viability is measured and serves as a key value to assess drug response. While methodology of the biological experiment quite similar to RNAi or Small Molecules Screening, its computational side brings new challenges when identifying new drug sets. Here drug set implies multiple drugs that cause similar response against single target (cancer sample).  In other words, Drug Set Enrichment Analysis intends to facilitate the assessment of drug responses across multiple cell lines and also allow for drug response prediction. The figure below shows how cell lines from different cancer types respond to treatment with Nilotinib, CYT387 and EKB.569 drugs.

A Figure goes here...




\section{Problem statement}

We can see different drug response from the picture, which means that some drugs perform better on a certain condition than the others. Thus, a set of drugs can be identified to act efficiently when curing a particular disease. Consequently, the first task would be to apply clustering to the samples in order to identify those drug sets. Then, once a new drug comes it should be classified and assigned to one of the existing clusters, which refers to Drug Set Enrichment Analysis. So, drug response prediction and classification appear as the second challenge in the project.

\section{Helpful hints}

The project will be conducted within the group of Individualized Systems Medicine in Institute for Molecular Medicine Finland FIMM.

\section{Structure of the Thesis}
\label{section:structure} 

You should use transition in your text, meaning that you should help
the reader follow the thesis outline. Here, you tell what will be in
each chapter of your thesis. 



\chapter{Background}
\label{chapter:background} 

Functional approach studies cell behaviour. In this case we are not looking at the genes that make up a canser, but we are focusing on the cancer itself. By studying the actual behaviour of the cells when exposed to the drug we can find a right drug regardless of what the gene test said. Cancer drugs usually target a scpecific gene mutation, however in practice there is a possibility that these drugs can work also for those patients who do not have the mutation at all.

Testing cancerous cells against drugs in vitro can double median survival time. These sensitivity asseys can also save a lot of resources.

\section{Drug Sensitivity and Restance Testing (DSRT)}

\section{What is missing?}

\section{Prospective Applications and usage}
\subsection{Direct Outcome}
\subsection{Further inferences}

\section{Literature Review}
\section{Relation to GSEA}


\input{3environment.tex}

\chapter{Methods}
\label{chapter:methods}

We have now stated the problem, and we are ready to do something
about it!  \emph{How} are we going to do that? What methods do we
use?  You also need to review existing literature to justify our
choices, meaning that why we have chosen the method to be applied in
the work.

\section{Preliminary Stage}
\subsection{DSRT Data Integration}
\subsection{Drug Clustering and Annotations}

\section{Enrichment Stage}
\subsection{Sensitive Drugs}
\subsection{Enrichment and Inference}


 
\chapter{Implementation}
\label{chapter:implementation}

You have now explained how you are going to tackle your problem. 
Go do that now! Come back when the problem is solved!



\input{6evaluation.tex}
 
\input{7discussion.tex}
 
\input{8conclusions.tex}


% Load the bibliographic references
% ------------------------------------------------------------------
% You can use several .bib files:
% \bibliography{thesis_sources,ietf_sources}
\bibliography{sources}


% Appendices go here
% ------------------------------------------------------------------
% If you do not have appendices, comment out the following lines
\appendix
\input{appendices.tex}

% End of document!
% ------------------------------------------------------------------
% The LastPage package automatically places a label on the last page.
% That works better than placing a label here manually, because the
% label might not go to the actual last page, if LaTeX needs to place
% floats (that is, figures, tables, and such) to the end of the 
% document.
\end{document}
